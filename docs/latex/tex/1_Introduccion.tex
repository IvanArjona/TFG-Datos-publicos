\capitulo{1}{Introducción}

La \textit{minería de datos} es una metodología para descubrir relaciones ocultas en grandes cantidades de datos utilizando técnicas de inteligencia artificial. Actualmente se está extendiendo bastante el uso de técnicas de minería de datos para realizar análisis sociológicos, como por ejemplo, cuándo se van a cometer crímenes incluso antes de que se cometan, como ocurría en la película `Minority Report' \cite{movie:minorityreport}, pero ahora ya es una realidad. Otro ejemplo es el reciente uso de \textit{big data} en China como un sistema disciplinario \cite{art:chinabigdata}. En ambos ejemplos destaca la importancia de los datos demográficos y sociológicos que hasta ahora a penas se han tratado en el ámbito español.

Un tema que está siendo investigado actualmente es comprender las causas por las que se producen crímenes. Por ello, se han realizado estudios para investigar las tasas de criminalidad utilizando técnicas de minería de datos \cite{art:crimeprediction} \cite{art:crimeprediction2} a partir de datos demográficos y sociológicos con el objetivo de determinar los lugares donde asignar más recursos para reducir estos índices de criminalidad.

Se dice que los datos se están convirtiendo en el nuevo petróleo, en el sentido de que con unos datos de buena calidad pueden descubrirse relaciones insospechadas hasta el momento. Una fuente importante de datos son los datos públicos, pero tiene inconvenientes, estos datos se obtienen de múltiples fuentes, por lo que cada una de ellas nos proporciona la información organizada de forma dirente y en distintos formatos, también podrían ser accesibles de modo diferente, lo que nos complica la laboral al analizarlos.

El objetivo de este trabajo es llevar estos estudios al ámbito español y unificar estas fuentes de datos de forma que el acceso sea uniforme, mucho más sencillo y permitiendo combinar fuentes de datos de forma directa. Se ha realizado integrando varias fuentes de datos públicas utilizando técnicas de web scraping (\ref{webscraping}). Para  después visualizar estos datos de manera sencilla, permitir crear mapas temáticos (\ref{mapascoropleticos}) y facilitar la exportación de los datos para después ser utilizados en procesos de minería de datos de forma directa.

Por último, agradecimiento a la idea original de Roberto Cuesta, capitán de la guardia civil y estudiante del doctorado de la Universidad de Burgos, con una tesis enfocada a la aplicación de técnicas de minería de datos en el ámbito criminológico.

\subsection{Fuentes de datos}

A continuación se van a listar las fuentes de datos de organismos públicos que se han integrado.

\begin{itemize}
	\tightlist
	\item
	Servicio Público de Empleo Estatal (SEPE)
	\begin{itemize}
		\tightlist
		\item
		Paro registrado por municipio\footnote{\href{https://datos.gob.es/catalogo/e00142804-paro-registrado-por-municipios}{https://datos.gob.es/catalogo/e00142804-paro-registrado-por-municipios}}.
		\item
		Contratos registrados por municipio\footnote{\href{http://datos.gob.es/es/catalogo/e00142804-contratos-por-municipios}{http://datos.gob.es/es/catalogo/e00142804-contratos-por-municipios}}.
		\item
		Demandantes de empleo por municipio\footnote{\href{http://datos.gob.es/es/catalogo/e00142804-demandantes-de-empleo-por-municipios}{http://datos.gob.es/es/catalogo/e00142804-demandantes-de-empleo-por-municipios}}.
	\end{itemize}
	\item 
	Estadísticas de la renta por municipio (Agencia tributaria)\footnote{\href{https://www.agenciatributaria.es/AEAT.internet/datosabiertos/catalogo/hacienda/Estadistica\_de\_los\_declarantes\_del\_IRPF\_por\_municipios.shtml}{https://www.agenciatributaria.es/AEAT.internet/datosabiertos/catalogo/hacienda/ Estadistica\_de\_los\_declarantes\_del\_IRPF\_por\_municipios.shtml}}.
	\item
	Instituto Nacional de Estadística (INE)
	\begin{itemize}
		\item
		Estadísticas de población por sexo y edad\footnote{\href{http://www.ine.es/jaxi/Tabla.htm?path=/t20/e245/p05/a2011/l0/\&file=00000001.px\&L=0}{http://www.ine.es/jaxi/Tabla.htm?path=/t20/e245/p05/a2011/l0/\&file=00000001.px\&L=0}}.
		\item
		Relación de municipios y códigos por provincias\footnote{\href{http://www.ine.es/daco/daco42/codmun/codmunmapa.htm}{http://www.ine.es/daco/daco42/codmun/codmunmapa.htm}}.
	\end{itemize}
\end{itemize}


\subsection{Material entregado}

Material adjunto a la memoria:

\begin{itemize}
	\item 
	Aplicación web en Flask.
	\item 
	Scripts para la integración de fuentes de datos.
	\item
	Scripts para la ejecución.
	\item 
	Aplicación de escritorio para juntar csv.
	\item 
	Anexos.
\end{itemize}

Recursos disponibles en internet:

\begin{itemize}
	\item 
	\foothref{Página web del proyecto}{https://tfg-datos-publicos.nanoapp.io/}.
	\item
	\foothref{Repositorio del proyecto}{https://github.com/IvanArjona/TFG-Datos-publicos}.
\end{itemize}
