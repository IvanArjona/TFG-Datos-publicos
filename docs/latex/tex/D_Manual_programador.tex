\apendice{Documentación técnica de programación}

\section{Introducción}

En este anexo se explica todo lo que tiene que conocer el programador tanto para instalar y ejecutar la aplicación como para poder seguir con el desarrollo.

\section{Estructura de directorios}

A continuación se van a mencionar los directorios con una breve descripción de su contenido para ayudar a comprender la estructura del proyecto. Además de los directorios, también se van a incluir los archivos más importantes. \\

\dirtree{%
	.1 /.
	.2 config/ \desc{Archivos de configuración}.
	.3 config.py \desc{Fichero de configuración de la base de datos}.
	.2 docs/ \desc{Documentación: memoria y anexos}.
	.3 latex/ \desc{Código en Latex para generar la documentación}.
	.4 img/ \desc{Imágenes de la documentación}.
	.4 tex/ \desc{Secciones de la documentación en Latex}.
	.3 memoria.pdf \desc{Memoria}.
	.3 anexos.pdf \desc{Anexos}.
	.2 fuentes/ \desc{Paquete para la carga de datos a partir de las fuentes. Un fichero por cada fuente de datos}.
	.2 prototipos/ \desc{Notebooks de IPython utilizados en un principio para implementar las fuentes de datos antes de pasarlo a la aplicación. Es más fácil trabajar con notebooks porque se puede ver el resultado en cada paso}.
	.2 test/ \desc{Pruebas de la aplicación}.
	.3 iterfaz/ \desc{Pruebas de interfaz hechas con \textit{Selenium}}.
	.3 unitarios/ \desc{Pruebas unitarias hechas con \textit{unittest}}.
	.2 web/ \desc{Raíz de la aplicación web hecha con \textit{Flask}}.
	.3 forms/ \desc{Formularios de \textit{Wtforms}}.
	.3 geojson/ \desc{Archivos GeoJSON que delimitan los límites geográficos en la representación de mapas}.
	.3 static/ \desc{Ficheros estáticos}.
	.4 css/ \desc{Hojas de estilo CSS}.
	.4 imagenes/ \desc{Imágenes}.
	.4 js/ \desc{Scripts para el comportamiento dinámico en \textit{Javascript}}.
	.3 templates/ \desc{Plantillas para generar HTML en \textit{Jinja2}}.
	.2 actualiza-fuentes.py \desc{Script para actualizar las fuentes de datos en la base de datos}.
	.2 boxfile.yml \desc{Archivo de configuración de Nanobox}.
	.2 joincsv.py \desc{Aplicación para juntar ficheros CSV}.
	.2 requirements.txt \desc{Dependencias de la aplicación (paquetes de Python)}.
	.2 run.py \desc{Script para ejecutar la aplicación web}.
}

\section{Manual del programador}

Esta sección tiene como objetivo explicar los puntos más importantes a tener en cuenta por desarrolladores que puedan seguir ampliando este proyecto.

\subsection{Añadir fuentes de datos}

El funcionamiento de esta aplicación se basa en los datos obtenidos de fuentes públicas, por lo que un punto importante es explicar cómo añadir más fuentes a nuestra aplicación.

La lógica de esta parte de la aplicación se encuentra en el paquete \textit{Fuentes}.

Lo primero que tenemos que hacer al añadir una fuente es crear un nuevo fichero con el nombre de la fuente. En este fichero crearemos una clase con el mismo nombre. Esta clase heredará de \textit{Fuente} (situado en \textit{Fuente.py}).

Esta clase tiene que implementar el método \textit{carga()}. La funcionalidad de este método es devolver un dataframe de Pandas a partir de los datos realizando una consulta a la fuente que se está implementando. Esta clase tiene que tener un constructor para parametrizar las diferentes consultas que se puedan realizar a la misma fuente de datos.

Por ejemplo, observamos que en el SEPE, la url para consultar el número de contratos en 2018 es la siguiente:

\begin{lstlisting}
https://sede.sepe.gob.es/es/portaltrabaja/resources/sede/
datos_abiertos/datos/Contratos_por_municipios_2018_csv.csv
\end{lstlisting}
Y la url para consultar los datos del paro en 2018 es la siguiente:

\begin{lstlisting}
https://sede.sepe.gob.es/es/portaltrabaja/resources/sede/
datos_abiertos/datos/Paro_por_municipios_2018_csv.csv
\end{lstlisting}

Observamos que el principio de la url es el mismo, por lo que lo guardaremos en una variable. Podemos observar que lo único que cambia es el nombre del fichero, en el que también aparece el año. De modo que podemos dividirlo como dos parámetros en el constructor: nombre de fichero y año. Además se incluirá la descripción, que depende de cada consulta.

\newpage

\begin{lstlisting}[caption=Ejemplo de fuente de datos (Sepe)]
class Sepe(Fuente):
	"""
	Fuente de datos para el Servicio Público de Empleo Estatal
	"""

	def __init__(self, url, anios, tabla, descripcion_):
		url_sepe = 'https://sede.sepe.gob.es/es/portaltrabaja/resources/sede/datos_abiertos/datos/'
		self.url = url_sepe + url
		self.anios = anios
		descripcion = descripcion_ + " Servicio Público de Empleo Estatal."
		super().__init__('sepe', tabla, descripcion)

    @staticmethod
	def procesa_datos(url):
		# Procesa los datos
		df = pd.read_csv(url, sep=';', encoding='latin1', header=1)
		return df

	def carga(self):
		dataframes = []
		for anio in self.anios:
			url = self.url.format(anio)
			df = self.procesa_datos(url)
			dataframes.append(df)
		df = pd.concat(dataframes)
		return df
\end{lstlisting}

Heredando de la clase \textit{Sepe} crearemos una subclase por cada una de las consultas que queramos realizar. En esta subclase añadiremos un constructor sin parámetros, en el que se llamará al constructor de su superclase parametrizándolo.

Por ejemplo, se crearía la subclase \textit{SepeContratos}, parametrizando el constructor del padre con el nombre del fichero y los años disponibles. Así quedaría la clase \textit{SepeContratos}:

\begin{lstlisting}[caption=Ejemplo de fuente de datos (SepeContratos)]
class SepeContratos(Sepe):
	"""
	Contratos por sexo y por edades por cada municipio
	"""
	
	def __init__(self):
		url = 'Contratos_por_municipios_{}_csv.csv'
		anios = range(2006, 2019)
		descripcion = 'Contratos por sexo y por edades.'
		super().__init__(url, anios, 'contratos', descripcion)
\end{lstlisting}

Una vez implementadas estas clases hay que añadirlas a la lista de fuentes de datos para que se carguen al ejecutar el script de actualizar fuentes. Esta lista está en el fichero \textit{\_\_init\_\_.py} del paquetes \textit{Fuentes}.

\begin{lstlisting}[caption=Lista con todas las fuentes]
from fuentes.aeat import AeatRenta
from fuentes.municipios import Municipios
from fuentes.ine import InePoblacion
from fuentes.sepe import SepeContratos, SepeEmpleo, SepeParo
from fuentes.mir import MirCongreso


# Lista de todas las fuentes disponibles
fuentes = [
	Municipios,
	AeatRenta,
	InePoblacion,
	SepeContratos,
	SepeEmpleo,
	SepeParo,
	MirCongreso
]
\end{lstlisting}

En el fichero \textit{Fuente.py} tenemos además dos decoradores que se pueden utilizar sobre la función \textit{carga()} (o cualquier otra función que devuelva un dataframe) para realizar ciertas acciones comunes.

\begin{itemize}
	\item @to\_numeric: convierte todos las columnas con datos numéricos almacenados como cadenas a tipos de datos numéricos.
	\item @rename(dict): renombra las columnas de un dataframe de Pandas a las pasadas en el diccionario. Siendo la clave el nombre actual y el valor el nuevo nombre.
\end{itemize}

\section{Instalación y ejecución del proyecto} \label{instalacionprogramador}

\subsection{Instalación}

\subsubsection{MongoDB}

Antes de empezar con la instalación de la aplicación tenemos que instalar la base de datos.
Se ha utilizado una base de datos no relacional MongoDB.

Concretamente se instaló la versión \textit{3.6.4 Community Server} \cite{misc:mongodb} para Windows.

También se ha utilizado \textit{MongoDB Compass} \cite{misc:mongodb} para visualizar el contenido de la base de datos. Esta herramienta es opcional.

\subsubsection{Python}

Esta aplicación se ha desarrollado utilizando la version 3.6.4 \cite{misc:python3}, por lo que se recomienda utilizar esta versión o una posterior. En cualquier caso debe de instalarse Python 3 o superior para evitar incompatibilidades.

Todos los paquetes utilizados en la aplicación están listados en ficheros \textit{requirements.txt} junto con sus correspondientes versiones.

Habrá que instalar todas las dependencias utilizando pip\footnote{Gestor de paquetes de Python \cite{wiki:pip}}:

\begin{lstlisting}
pip install -r requirements.txt
\end{lstlisting}

\subsection{Ejecución}

\subsection{Configuración}

Para configurar los datos de acceso a la base de dato podemos utilizar las variables de entorno: \textit{DATA\_DB\_HOST}, \textit{DATA\_DB\_PORT}, \textit{DATA\_DB\_NAME}, \textit{DATA\_DB\_USER}, \textit{DATA\_DB\_PASS}.

En caso de usar Nanobox estas variables estarán configuradas por defecto, por lo que no tendríamos que hacer nada.

Si dejamos los valores por defecto al instalar MongoDB, tampoco hará falta hacer nada, se cogerán estos valores automáticamente.

El el archivo \textit{config/config.py} también podemos modificar esta configuración. Si se añade alguna característica más en el futuro que requiera algún tipo de configuración debería añadirse una variable a esta clase.

\subsubsection{Actualización de la base de datos}

Antes de ejecutar la aplicación tendremos que construir la base de datos con el contenido de todas nuestras fuentes por primera vez.

Para ello simplemente hay que ejecutar el fichero \textit{actualiza-fuentes.py}. Puede tardar un rato debido a la gran cantidad de datos que se van a cargar.

Este paso puede repetirse cada vez que se quiera actualizar las fuentes de datos. Puede ser ejecutado una vez al mes para mantener al día las fuentes con datos mensuales.

\begin{lstlisting}
python actualiza-fuentes.py
\end{lstlisting}

\subsubsection{Servidor web}

El servidor Flask ya se ha instalado como un paquete, por lo que no necesita más instalación.
Para ejecutarlo hay un script \textit{run.py} que nos lanza el servidor en el puerto 5000.

Una vez lanzado podremos acceder a la página web desde \href{http://localhost:5000}{localhost:5000}.

\begin{lstlisting}
python run.py
\end{lstlisting}

\section{Despliegue}

Para desplegar la aplicación se ha optado utilizar como servidor en la nube DigitalOcean y Nanobox como microservicio para facilitarnos la instalación de la máquina en la nube y la configuración del servidor.

\subsection{Instalación}

\subsubsection{DigitalOcean}

Primero tendremos que registrarnos en \foothref{DigitalOcean}{https://www.digitalocean.com/} y obtener el \textit{token} de nuestro usuario. Tendremos que poner una tarjeta de crédito de la que nos cobrarán el importe del servidor.

Podría utilizarse otro proveedor de hosting como \textit{Amazon Web services} o \textit{Google Compute}.

\subsubsection{Nanobox}

Después nos registraremos en \foothref{Nanobox}{https://nanobox.io/} y creamos una nueva aplicación utilizando el token que hemos obtenido antes en DigitalOcean.

Ahora elegiremos el plan que queramos contratar. Para este proyecto será suficiente con el plan más basico de \EUR{5}, 1 CPU y 1GB de ram.

Tras crear la aplicación instalamos el \foothref{cliente de nanobox}{https://docs.nanobox.io/install/}. La primera vez que lancemos un comando nos pedirá los datos para iniciar sesión en nuestra cuenta.

En el fichero \textit{nanobox.yml} tenemos la configuración con los componentes que se van a instalar y los comandos que se ejecutan al desplegar la aplicación.

\subsection{Despliegue}


Una vez todo instalado y configurado, cada vez que queramos actualizar la aplicación del servidor nos situamos en la carpeta del proyecto y lanzamos el siguiente comando:

\begin{lstlisting}
nanobox deploy
\end{lstlisting}

Con esto, de forma transparente para nosotros, se crea una máquina virtual con nuestro proyecto en la que se instalan la base de datos, el entorno de ejecución y los paquetes y se ejecuta la aplicación en el servidor.

\section{Pruebas del sistema}

Se ha realizado una batería de pruebas para verificar el correcto funcionamiento de la aplicación.

\subsection{Pruebas unitarias}

Se han realizado pruebas de cada módulo para comprobar su funcionamiento. Estas pruebas están situadas en la carpeta \textit{/test/unitarios} y se han realizado con el framework de pruebas \textit{unittest} \cite{misc:unittest}.

Estas pruebas se han centrado en la ejecución de todos los métodos utilizados en la consulta (\textit{consulta.py}) y en la representación de mapas (\textit{mapas.py}).

\textit{Unittest} está integrado en la librería estándar de Python, por lo que no hará falta instalar ningún paquete adicional. 

Para ejecutar los test utilizamos el siguiente comando en la consola:

\begin{lstlisting}
python -m unittest discover test.unitarios
\end{lstlisting}

\subsection{Pruebas de interfaz}

Para probar la interfaz web se ha utilizado \textit{Selenium} con la API que proporciona para Python \cite{misc:seleniumpython}, en lugar de grabar las acciones mediante el plugin de Firefox.

Estas pruebas están situadas en la carpeta \textit{/test/interfaz} y se ejecutan de manera similar a las pruebas unitarias, ya que también están hechas con el framework \textit{unittest}.

Para poder ejecutar estas pruebas primero tenemos que instalar el paquete de Selenium:

\begin{lstlisting}
pip install selenium
\end{lstlisting}

Una vez instalado podemos ejecutar las pruebas con el siguiente comando:

\begin{lstlisting}
python -m unittest discover test.interfaz
\end{lstlisting}

\subsection{Todas las pruebas}

También podemos ejecutar todas las pruebas al mismo tiempo:

\begin{lstlisting}
python -m unittest discover
\end{lstlisting}