\capitulo{2}{Objetivos del proyecto}

Este trabajo pretender colocar la primera piedra en un proyecto mucho mayor y más ambicioso.

El objetivo es desarrollar un sistema que permita integrar múltiples fuentes de datos para trabajar con ellas de manera unificada. Y sentar las bases para que, en el futuro, se puedan integrar más datos, formas de visualizarlos y procesarlos.

El usuario objetivo de esta aplicación es el científico de datos que quiera encontrar patrones entre datos públicos a nivel  municipal. Los usuarios podrán ser académicos o también periodístas, políticos o servicios públicos.

Este apartado explica de forma precisa y concisa cuales son los objetivos que se persiguen con la realización del proyecto. Se puede distinguir entre los objetivos marcados por los requisitos del software a construir y los objetivos de carácter técnico que plantea a la hora de llevar a la práctica el proyecto.

\section{Objetivos funcionales}

\begin{itemize}
	\item Integrar varias fuentes de datos públicas en una única base de datos. Estos datos son datos de carácter sociológicos, económicos y demográfico a nivel municipal en España.
	\item Permitir añadir nuevos conjunto de datos de forma sencilla.
	\item Desarrollar un algoritmo para la carga de varias fuentes de datos en una base de datos de manera automatizada, de forma que el administrador pueda añadir nuevos conjuntos de datos de forma sencilla.
	\item Desarrollo de una aplicación web que permita la consulta de los datos de manera sencilla y visual.
	\item Facilitar la interpretación de los datos utilizando un mapas coropléticos interactivo.
	\item Desplegar la aplicación web en un servidor de forma que sea fácil de actualizar cada vez que se realice un cambio. Además de funcionar en un entorno local.
\end{itemize}

\section{Objetivos técnicos}

\begin{itemize}
	\item Seguimiento de principios de desarrollo ágiles durante el desarrollo del proyecto.
	\item Utilizar \textit{Flask} como framework para desarrollar la aplicación web utilizando la arquitectura \textit{Modelo Vista Controlador}.
	\item Utilizar \textit{git} como sistema de control de versiones, junto con \textit{Github} y \textit{Zenhub} para la organización del proyecto.
	\item Utilizar herramientas para mejorar la calidad del código como \textit{codacy}.
	\item Manejar bases de datos no relacionales \textit{NoSQL}.
	\item Elegir herramientas de \textit{software libre} siempre que sea posible.
	\item Proporcionar una instalación sencilla para el desarrollador y una interfaz visual y fácil de utilizar para el usuario.
\end{itemize}


