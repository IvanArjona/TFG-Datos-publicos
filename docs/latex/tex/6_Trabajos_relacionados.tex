\capitulo{6}{Trabajos relacionados}

%TODO: papers

\section{Mapa del paro}

\foothref{Mapa del paro}{http://mapadelparo.com/} es una página web que nos permite visualizar el paro por municipios, provincias y comunidades en un mapa coroplético. Ver figura \ref{fig:relacionados/mapadelparo}.

Los datos del paro contenidos en esta página web también están en este proyecto, con la ventaja de que además se pueden combinar con otros datos públicos y se puede descargar la información.

\imagen{relacionados/mapadelparo}{Paro por comunidades autónomas}

\section{Ciudatos}

\foothref{Ciudatos}{http://www.ciudatos.com/visualizacion} nos permite visualizar varios datos públicos de Colombia mediante gráficas de barras y posicionarlos en el mapa. Ver figura \ref{fig:relacionados/ciudatos}.

Esta herramienta no nos permite visualizar los datos con mapas coropléticos, simplemente sitúa los gráficos de barras de cada región en su correspondiente sitio.

Como ventaja nuestra aplicación tiene datos de España y permite visualizar varios datos diferentes entre ellos al mismo tiempo.

\imagen{relacionados/ciudatos}{Bienestar por ciudades (Colombia)}

\section{Eurostat}

Otra representación de datos públicos en mapas es \foothref{Eurostat}{http://ec.europa.eu/eurostat/en/web/lfs/statistics-illustrated}.

En esta representación se muestra en el mapa los porcentajes de empleo en cada país de Europa. Ver figura \ref{fig:relacionados/eurostat}.

Estos datos se escapan del alcance al que se quería llegar en este trabajo, pero podría considerarse para integrarlos en algún proyecto futuro.

\imagen{relacionados/eurostat}{Porcentaje de empleo por país (Europa)}
