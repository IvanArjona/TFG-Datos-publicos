\capitulo{6}{Trabajos relacionados}

Este capítulo está divido en dos secciones. Primero, se explica brevemente artículos de investigación de minería de datos relacionados con este proyecto. En la segunda sección se comparan algunas páginas web que visualizan datos públicos con este proyecto.

\section{Artículos de investigación}

\subsection{Crime prediction through urban metrics and statistical learning}

En este artículo se estudian las causas que pueden aumentar el índice de criminalidad utilizando técnicas de minería de datos. Para ello se utiliza un \textit{Random Forest} para predecir el número de homicidios en una ciudad a partir de datos sociológicos y demográficos. \cite{art:crimeprediction}

Como resultado se obtiene un 97\% de precisión utilizando el coeficiente de determinación $R^{2}$ \cite{wiki:r2} en la predicción de índices de criminalidad y se ordenan los indicadores que causan estos crímenes en función de su importancia.

\subsection{Detecting and investigating crime by means of data mining: a general crime matching framework}

En este artículo, al igual que el anterior, se utilizan técnicas de minería de datos para identificar las características que pueden llevar a cometer un crimen \cite{art:crimeprediction2}.

En esta aproximación se toma como entrada informes policiales narrativos escritos en texto plano. De estos textos se extraen las características para determinar la gravedad del delito utilizando análisis de grupos mediante \textit{clustering}.

\section{Páginas web}

\subsection{Mapa del paro}

\foothref{Mapa del paro}{http://mapadelparo.com/} es una página web que nos permite visualizar el paro por municipios, provincias y comunidades en un mapa coroplético. Ver figura~\ref{fig:relacionados/mapadelparo}.

Los datos del paro contenidos en esta página web también están en este proyecto, con la ventaja de que además se pueden combinar con otros datos públicos y se puede descargar la información.

\imagen{relacionados/mapadelparo}{Paro por comunidades autónomas}

\subsection{Ciudatos}

\foothref{Ciudatos}{http://www.ciudatos.com/visualizacion} nos permite visualizar varios datos públicos de Colombia mediante gráficas de barras y posicionarlos en el mapa. Ver figura \ref{fig:relacionados/ciudatos}.

Esta herramienta no nos permite visualizar los datos con mapas coropléticos, simplemente sitúa los gráficos de barras de cada región en su correspondiente sitio.

Como ventaja nuestra aplicación tiene datos de España y permite visualizar datos resultantes de juntar varias subconsultas.

\imagen{relacionados/ciudatos}{Bienestar por ciudades (Colombia)}

\subsection{Eurostat}

Otra representación de datos públicos en mapas es \foothref{Eurostat}{http://ec.europa.eu/eurostat/en/web/lfs/statistics-illustrated}.

En esta representación se muestra en el mapa los porcentajes de empleo en cada país de Europa. Ver figura \ref{fig:relacionados/eurostat}.

Estos datos se escapan del alcance al que se quería llegar en este trabajo, pero podría considerarse para integrarlos en algún proyecto futuro.

\imagen{relacionados/eurostat}{Porcentaje de empleo por país (Europa)}

\subsection{Comparación}

Por último, se en la tabla \ref{tabla:comparacionwebs}, se ha realizado una comparación entre las características que ofrece este proyecto y las páginas web relacionadas.

\tablaSmall{Comparativa con herramientas de visualización de datos}{l c c c c}{comparacionwebs}
{ \multicolumn{1}{l}{Herramientas} & TFG & Mapa del paro & Ciudatos & Eurostat \\}{ 
	Exportación & \cmark & \xmark & \cmark & \cmark \\
	Filtro de datos & \cmark & \xmark & \cmark & \cmark \\
	Múltiples fuentes & \cmark & \xmark & \cmark & \xmark \\
	Columnas calculadas & \cmark & \xmark & \xmark & \xmark \\
	Mapas interactivos & \cmark & \cmark & \cmark & \cmark \\
	Región & España & España & Colombia & Europa \\
	Nivel & \makecell{Provincias\\Municipios} & \makecell{Comunidades\\Provincias\\Municipios} & \makecell{Ciudades} & Países \\

} 