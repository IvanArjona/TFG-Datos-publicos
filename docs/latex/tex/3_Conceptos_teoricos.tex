\capitulo{3}{Conceptos teóricos}

En aquellos proyectos que necesiten para su comprensión y desarrollo de unos conceptos teóricos de una determinada materia o de un determinado dominio de conocimiento, debe existir un apartado que sintetice dichos conceptos.


\section{Bases de datos no relacionales}

Las bases de datos no relacionales\cite{wiki:nosql}, también llamadas NoSQL (`Not Only SQL') son bases de datos optimiazadas para ser utilizadas con modelos de datos sin esquema y potencialmente escalables.

A diferencia de las bases de datos relacionales, aquí no hay tables, esquemas ni relaciones, sino que los datos pueden almacenarse con cualquier esquema sin tener que seguir todos la misma estructura.

En este tipo de bases de datos tenemos colecciones en lugar de tablas y dentro de estas colecciones tenemos documentos. Estos documentos pueden contener cualquier información y se almacenan siguiendo un esquema json.

Una de las razones importantes por las que usar bases de datos NoSQL en lugar de relacionales es la alta velocidad de consulta y la gran capacidad de escalabilidad.


\section{Web scraping}

Web Scraping \cite{wiki:webscraping} es una técnica para extraer información de sitios web directamente de su código fuente sin utilizar APIs\footnote{Interfaz de programación de aplicaciones} proporcionadas por el propio sitio.

Lo que hacemos con un \textit{web scraper} es buscar información dentro de un documento web siguiendo ciertos patrones en la estructura de su código fuente y extraer esta información a nuestro entorno local.

Las técnicas de web scraping se centran en transformar datos sin estructura de una página web en datos estructurados para poder ser almacenados y analizados posteriormente. Por ejemplo, en una base de datos, hojas de cálculo de dataframes de pandas.

\section{Mapas coropléticos}

Un mapa coroplético \cite{wiki:mapascoropleticos} es un mapa topológico dividido en regiones en el que cada una de estas regiones se pinta de un color de acuerdo a una medida estadística.

Para ilustrarlo mejor pondremos el ejemplo del siguiente mapa en la figura \ref{fig:conceptos/mapa-paro}. En él se compara el paro de todas las provincias de España pintando con colores más cálidos las provincias con mayor porcentaje de paro y con colores más fríos las provincias con menor porcentaje de paro.

\imagencite{conceptos/mapa-paro}{Mapa coroplético del paro en las provincias españolas}{art:mapaparo}

