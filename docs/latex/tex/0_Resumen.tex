% Abstract en castellano
\renewcommand*\abstractname{Resumen}
\begin{abstract}
En este proyecto se va a desarrollar una aplicación web que permita la consulta de datos demográficos y sociológicos de fuentes de datos públicas en el territorio español, permitiendo combinar estas fuentes y realizar consultas más avanzadas mediante columnas calculadas.

Se trata de representar estos datos, visualizarlos en mapas temáticos sobre el mapa de España a nivel de provincia y municipio, permitiendo su exportación a \textit{CSV} y \textit{JSON} para su uso en procesos de minería de datos.

Los datos procesados se almacenará de forma local en una base de datos \textit{NoSQL} para acceder más rápido y no hacer llamadas constantemente a las fuentes de datos.

La aplicación web está disponible a través de la página del proyecto:

\href{https://tfg-datos-publicos.nanoapp.io/}{https://tfg-datos-publicos.nanoapp.io/}

\end{abstract}

\renewcommand*\abstractname{Descriptores}
\begin{abstract}
Aplicación web, bases de datos NoSQL, datos públicos, estudio sociológico, mapas temáticos, minería de datos, visualización.
\end{abstract}

\clearpage

% Abstract en inglés
\renewcommand*\abstractname{Abstract}
\begin{abstract}
This projects consist in the development of a web application to consult demographic and sociological data from public data sources in Spanish territory, allowing the combination of this sources and making more advanced queries through calculated columns.

The aim is to represent these data, visualize it in thematic maps on the map of Spain at the province and municipality level, allowing their exportation to \textit{CSV} and \textit{JSON} for data mining uses.

The processed data will be stored locally in a \textit{NoSQL} database to access faster and not make constantly calls to the data sources.

The web application is available through the project page:

\href{https://tfg-datos-publicos.nanoapp.io/}{https://tfg-datos-publicos.nanoapp.io/}

\end{abstract}

\renewcommand*\abstractname{Keywords}
\begin{abstract}
Data mining, NoSQL databases, public data, sociological study, thematic maps, visualization, web application.
\end{abstract}