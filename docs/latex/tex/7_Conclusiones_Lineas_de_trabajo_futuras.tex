\capitulo{7}{Conclusiones y Líneas de trabajo futuras}

En este capítulo se exponen las conclusiones derivadas del desarrollo del proyecto y comentarios para continuar con el trabajo.

\section{Conclusiones}

Tras haber acabado con la codificación del proyecto, creo que se han cumplido los objetivos que se habían planteado al comenzar con el proyecto. Como resultado se ha obtenido una herramienta que puede ser muy útil para ser utilizada en procesos de minería de datos que requieran datos demográficos y sociológicos en el ámbito español.

A nivel personal se han adquirido conocimientos de como utilizar \textit{bases de datos no relacionales}, de las que no se había tratado nada en la carrera y me interesaba aprender.

También se han ampliados los conocimientos en el desarrollo de páginas web, tanto en la parte de \textit{front-end} con el diseño y funcionamiento dinámico de la página, como en la parte de \textit{back-end} con la codificación en una aplicación con Python, sin experiencia hasta ahora.

Otro tema tratado del que no conocía hasta ahora es la utilización y configuración de \textit{servidores en la nube}, en este caso se han utilizado para desplegar la página web.

Puedo concluir que estoy satisfecho con el trabajo realizado en este proyecto, la experiencia ha sido muy positiva y he aprendido mucho sobre temas de los que conocía poco hasta ahora.

\section{Líneas de trabajo futuras}

Este trabajo es el inicio de un proyecto más completo y más ambicioso que se pretende continuar en trabajos de fin de grado futuros. A continuación se muestra una lista de posibles tareas a realizar en trabajos futuros:

\begin{itemize}
	\item Mejoras las columnas calculadas:
	\begin{itemize}
		\item Validar si la consulta es correcta dinámicamente con \textit{Javascript}.
		\item Permitir consultar más de una columna calculada.
		\item Permitir elegir el nombre de las columnas calculadas.
		\item Mejorar el autocompletado al introducir operadores.
	\end{itemize}
	\item Hacer que los datos estén disponibles en la web semántica siguiendo la descripción de RDF~\cite{misc:informationserviceengineering}.
	\item Mejorar la visualización de mapas: mostrar el valor de los atributos al pasar el ratón sobre una región, hacer que funcione el renderizado de mapas en Chrome. Puede que haya que cambiar de framework para representar mapas. 
	\item Integrar más fuentes de datos estatales. Como por ejemplo:
	\begin{itemize}
		\item Matrimonios por provincia de residencia del matrimonio, mes de celebración y forma de celebración del matrimonio
		\footnote{\href{http://www.ine.es/dynt3/inebase/index.htm?padre=3406}{http://www.ine.es/dynt3/inebase/index.htm?padre=3406}}.
		\item Balances de criminalidad
		\footnote{\href{http://www.interior.gob.es/prensa/balances-e-informes/2017}{http://www.interior.gob.es/prensa/balances-e-informes/2017}}.
		\item Estadísticas catastrales
		\footnote{\href{http://www.catastro.minhap.es/esp/estadisticas\_2.asp}{http://www.catastro.minhap.es/esp/estadisticas\_2.asp}}.
		\item Información meteorológica
		\footnote{\href{http://www.aemet.es/es/datos\_abiertos/catalogo}{http://www.aemet.es/es/datos\_abiertos/catalogo}}.
	\end{itemize}
	\item Integrar fuentes de datos de empresas privadas:
	\begin{itemize}
		\item Tendencias de búsqueda como \textit{Google Trends}
		\footnote{\href{https://trends.google.es/trends/?geo=ES}{https://trends.google.es/trends/?geo=ES}}.
		\item Porcentajes de audiencia por municipio (televisión, radio, periódico).
	\end{itemize}
\end{itemize}