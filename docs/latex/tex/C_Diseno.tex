\apendice{Especificación de diseño}

\section{Introducción}

En este anexo se explican los diseños que se han llevado a cabo para realizar los objetivos anteriores. Se incluye el diseño de datos, diseño procedimental y diseño arquitectónico.

\section{Diseño de datos}

La base de datos utilizada en este proyecto es una base de datos de tipo NoSQL, por lo que no hay un sistema relacional de tablas.

Como en la aplicación se trabaja con datos públicos, no se consideró realizar un sistema de usuarios, por lo que cualquiera puede usarla.

\subsection{Fuentes de datos}

Las fuentes de datos se almacenan cada una de ellas en una colección con sus campos correspondientes.

Todas estas fuentes de datos tiene dos campos comunes: código de municipio y código de comunidad. Estos campos se utilizan para agregar varias fuentes de datos en la consulta y dibujar los mapas coropléticos por provincia o municipio.

% TODO Diagrama entidad relación?

\subsection{Fronteras geográficas}

Para poder dibujar los mapas coropléticos es necesario tener almacenados los límites geográficos para pintar las porciones correspondientes.

Estos límites geográficos están almacenados an archivos \textit{geojson} \cite{misc:geojson}, que son estructuras siguiendo el formato \textit{json} para representar elementos geográficos sencillos.

Los geojson utilizados se han obtenido de \cite{misc:limitesmunicipales} en el caso de los límites geográficos por provincias y \cite{misc:carto} para los límites municipales.

En ambos casos se ha utilizado \foothref{mapshaper}{http://mapshaper.org/} \cite{misc:mapshaper} para minimizar los archivos y que el renderizado sea menos pesado en el navegador.

\section{Diseño procedimental}

\section{Diseño arquitectónico}


