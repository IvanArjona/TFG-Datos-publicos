\capitulo{4}{Técnicas y herramientas}

En esta sección se van a explicar, por un lado, las técnicas que se han segueoshaperido durante el desarrollo del proyecto y por otro, las herramientas utilizadas, que se han dividido en herramientas de desarrollo, gestión y documentación.

\section{Técnicas}

\subsection{Metodología ágil}

Este proyecto se ha realizado siguiendo los principios del manifiesto ágil~\cite{art:agilemanifesto}. En concreto se ha seguido la metodología \textit{scrum} \cite{book:scrum} con sprints normalmente de una semana y reuniones semanales de planificación y revisión.

Al utilizar esta metodología se persigue utilizar una estrategia de desarrollo incremental, en la que al final de cada sprint se tiene como resultado un incremento del software entregable. También se permite que los requisitos cambien a lo largo del proyecto, en lugar de fijarse al principio, consiguiendo un software de mejor calidad.

Un software creado con unos requisitos claros y fijos desde el principio puede ser un software de muy alta calidad y podría realizarse con una metodología tradicional, pero para proyectos exploratorios como este, en los que los requisitos pueden cambiar, es mejor utilizar una metodología ágil.

Para seguir esta metodología se han utilizado las \foothref{issues de Github}{https://github.com/IvanArjona/TFG-Datos-publicos/issues} (\ref{github}) y los tableros Kanban de \foothref{ZenHub}{https://www.zenhub.com/} (\ref{zenhub}) para organizar la pila de tareas a realizar en cada sprint.

\subsection{DevOps}

DevOps \cite{misc:devops} es una metodología para la creación de software en la que se integra el desarrollo de software y la administración de sistemas.

Se ha elegido esta metodología para conseguir desplegar rápidamente todos los cambios en el servidor web. Apoyándonos en herramientas como Nanobox (\ref{nanobox}) para facilitar este proceso.

\section{Herramientas}

\subsection{Herramientas de desarrollo}

A continuación se van a explicar las herramientas utilizadas para desarrollar el proyecto.

\subsubsection{MongoDB}

\begin{itemize}
	\tightlist
	\item
	Herramientas consideradas:
	\href{http://basho.com/riak/}{Riak}, 
	\href{http://cassandra.apache.org/}{Cassandra},
	\href{https://www.mongodb.com/}{MongoDB}, 
	\href{http://leveldb.org/}{LevelDB}.
	\item
	Herramienta elegida:
	\href{https://www.mongodb.com/}{MongoDB}.
\end{itemize}

\foothref{MongoDB}{https://www.mongodb.com/} \cite{misc:mongodb} es un sistema de bases de datos NoSQL. En esta herramienta los datos se guardan en forma de documentos con un esquema similar a JSON. Con este sistema se consigue una consulta de datos más rápida. Además se ha usado PyMongo \cite{docs:pymongo} como herramienta para integrar MongoDB en Python.

Tanto Riak como Cassandra son también bases de datos NoSQL, se descartaron porque no ofrecen soporte para equipos con sistema operativo Windows y no son bases de datos de documentos. 

LevelDB es una base de datos NoSQL de pares clave-valor, esta herramienta se descartó porque no se cree conveniente utilizar pares clave-valor para un proyecto como este y no hay tantos ejemplos en la documentación como en las otras herramientas.

\subsubsection{DigitalOcean} \label{digitalocean}

\begin{itemize}
	\tightlist
	\item
	Herramientas consideradas:
	\href{https://www.heroku.com/}{Heroku}, 
	\href{https://www.pythonanywhere.com/}{PythonAnywhere},
	\href{https://www.digitalocean.com/}{DigitalOcean}, 
	\href{https://aws.amazon.com/es/}{Amazon Web Services}.
	\item
	Herramienta elegida:
	\href{https://www.digitalocean.com/}{DigitalOcean}.
\end{itemize}

\foothref{DigitalOcean}{https://www.digitalocean.com/} es un proveedor de servidores privados, por ello podemos hacer lo que queramos con el servidor sin limitaciones más allá de la capacidad de procesamiento y memoria RAM. Se ha elegido este servicio porque nos da total libertad y se puede probar gratuitamente con \foothref{GitHub Education}{https://education.github.com/}~\cite{docs:digitalocean}.

Una alternativa que se consideró y de hecho, se probó es Heroku, en este caso se instala el entorno necesario de forma automática. El problema es que la base de datos en la capa gratuita sólo puede pesar 500MB como máximo y no es suficiente para este proyecto.

PythonAnywhere es un hosting para aplicaciones web en Python, el problema con este proveedor es que no ofrece bases de datos locales, por lo que habría que utilizar una remota. La única gratuita que se ha encontrado es \foothref{mLab}{https://mlab.com/}, la misma que usa Heroku, por lo que volvemos al mismo problema del límite de tamaño.

Por último, se consideró utilizar una instancia de \foothref{Amazon AWS EC2}{https://aws.amazon.com/es/ec2/}. Es muy similar a DigitalOcean, se eligió el primero porque es más sencillo de utilizar.

\subsubsection{Nanobox} \label{nanobox}
\begin{itemize}
	\tightlist
	\item
	Herramientas consideradas:
	\href{https://www.heroku.com/}{Heroku}, 
	\href{https://nanobox.io/}{Nanobox}.
	\item
	Herramienta elegida:
	\href{https://nanobox.io/}{Nanobox}.
\end{itemize}

\foothref{Nanobox}{https://nanobox.io/} es una herramienta que nos permite desplegar nuestra aplicación sin centrarnos en la infraestructura del servidor \cite{docs:nanobox}.

Para ello enlazamos nuestra cuenta de un proveedor en la nube (en este caso DigitalOcean) y nanobox se encargará de instalar el sistema operativo, configurarlo, instalar nuestra aplicación y sus dependencias y ejecutarla.

Heroku es un servicio muy similar, con la diferencia de que no podemos utilizar servidores en la nube externos, se descartó por lo ya explicado en el punto anterior.

\subsubsection{Flask}

\begin{itemize}
	\tightlist
	\item
	Herramientas consideradas:
	\href{http://flask.pocoo.org/}{Flask}, 
	\href{https://www.djangoproject.com/}{Django}.
	\item
	Herramienta elegida:
	\href{http://flask.pocoo.org/}{Flask}.
\end{itemize}

Como uno de los objetivos del proyecto es el de crear una página web, se han considerado varios frameworks web para Python. Entre ellos se ha seleccionado Flask.

\foothref{Flask}{http://flask.pocoo.org/} es un framework fácil de utilizar y muy flexible. No nos fuerza a utilizar una metodología específica y podemos organizar la aplicación con la estructura que queramos (a diferencia de \textit{Django}) \cite{book:flask}.

Además se incluyen herramientas para desplegar el servidor de desarrollo, para realizar pruebas de la aplicación y para hacer \textit{APIs REST}.

\subsubsection{Folium} \label{folium}

\begin{itemize}
	\tightlist
	\item
	Herramientas consideradas:
	\href{https://plot.ly/python/maps/}{Plotly maps}, 
	\href{http://python-visualization.github.io/folium/}{Folium},
	\href{https://leafletjs.com/}{Leaflet}.
	\item
	Herramienta elegida:
	\href{http://python-visualization.github.io/folium/}{Folium}.
\end{itemize}

Un aspecto importante de este proyecto es la representación de datos en el mapa. \\

Para ello se han considerado varias herramientas, de las cuales se ha elegido \foothref{Folium}{http://python-visualization.github.io/folium/} \cite{docs:folium}. Este paquete nos permite visualizar datos manipulados con Python y visualizarlos como mapas utilizando para ello \textit{LeafletJS} \cite{docs:leaflet}.

Se ha elegido esta herramienta porque nos permite representar mapas coropléticos (\ref{mapascoropleticos}) y mapas con clusters de datos \cite{misc:foliumcluster} utilizando estructuras de datos geográficas personalizadas (\textit{GeoJSON}) \cite{docs:geojson}.

\subsubsection{Dynatable}

\begin{itemize}
	\tightlist
	\item
	Herramientas consideradas:
	\href{https://www.dynatable.com/}{Dynatable}, 
	\href{https://datatables.net/}{Datatables}.
	\item
	Herramienta elegida:
	\href{https://www.dynatable.com/}{Dynatable}.
\end{itemize}

\foothref{Dynatable}{https://www.dynatable.com/} es un framework de javascript para visualizar tablas de una manera más clara y ordenada \cite{docs:dynatable}.

Este framework nos permite paginar los resultados, ordenar por alguna columna y buscar por cualquier campo dentro de la tabla. Todo esto sin tener que recargar la página.

\subsubsection{Bootstrap}

\begin{itemize}
	\tightlist
	\item
	Herramientas consideradas:
	\href{https://getbootstrap.com/}{Bootstrap}, 
	\href{https://foundation.zurb.com/}{Foundation}.
	\item
	Herramienta elegida:
	\href{https://getbootstrap.com/}{Bootstrap}.
\end{itemize}

\foothref{Bootstrap}{https://getbootstrap.com/} es un framework para desarrollar páginas web con HTML, CSS y javascript para facilitar el diseño de páginas responsive, que se adapten al tamaño de la pantalla del dispositivo.

Bootstrap nos ofrece una serie de clases de CSS predefinidas por lo que podemos crear la web de proyecto prácticamente sin tocar hojas de estilos CSS. Otra característica es que dispone de funciones de javascript para llamar a eventos de ciertos elementos, como puede ser al pulsar un item de un menú desplegable.

\subsubsection{PyCharm}

\begin{itemize}
	\tightlist
	\item
	Herramientas consideradas:
	\href{https://code.visualstudio.com/}{Visual Studio Code}, 
	\href{https://www.jetbrains.com/pycharm/}{PyCharm}.
	\item
	Herramienta elegida:
	\href{https://www.jetbrains.com/pycharm/}{PyCharm}.
\end{itemize}

\foothref{PyCharm}{https://www.jetbrains.com/pycharm/} es un IDE (\textit{Entorno de Desarrollo Integrado}) para Python basado en IntelliJ. Posee herramientas para ayudarnos con el desarrollo como el autocompletado y herramientas para refactorizar de forma automática.

Se ha considerado porque, a diferencia de otros IDEs similares, tiene soporte para Flask y MongoDB, lo que nos facilita programar más rápidamente.

En este proyecto se ha utilizado la versión \textit{Professional}, pero con la versión \textit{Community} gratuita sería suficiente.

\subsection{Herramientas de gestión}

A continuación se van a describir las herramientas que se han utilizada gestionar el proyecto.

\subsubsection{Github} \label{github}

\begin{itemize}
	\tightlist
	\item
	Herramientas consideradas:
	\href{https://github.com/}{Github}, 
	\href{https://bitbucket.org/product}{BitBucket},
	\href{https://about.gitlab.com/}{GitLab}.
	\item
	Herramienta elegida:
	\href{https://github.com/}{Github}.
\end{itemize}

\foothref{GitHub}{https://github.com/} es un servicio web para alojar proyectos utilizando un sistema de control de versiones \textit{Git}.

Se ha utilizado esta herramienta para alojar el \href{https://github.com/IvanArjona/TFG-Datos-publicos}{repositorio del proyecto} llevando un registro de todos los cambios realizados desde el inicio.

Además se han utilizado las \textit{Issues} y \textit{Milestones} de GitHub para planificar las tareas a realizar en cada sprint.

\subsubsection{ZenHub} \label{zenhub}

\begin{itemize}
	\tightlist
	\item
	Herramientas consideradas:
	\href{https://help.github.com/articles/about-project-boards/}{Github Projects}, 
	\href{https://www.zenhub.com/}{ZenHub},
	\href{https://www.gitkraken.com/glo}{GitKraken Glo}.
	\item
	Herramienta elegida:
	\href{https://www.zenhub.com/}{ZenHub}.
\end{itemize}

\foothref{ZenHub}{https://www.zenhub.com/} es una herramienta de gestión de proyectos integrada en Github mediante una extensión para el navegador. Nos proporciona un tablero cambas sobre el que mover las \textit{issues} entre varios estados.

Las tareas se pueden estimar estableciendo un número `\textit{Story points}' basados en la serie de Fibonacci en función del trabajo que se cree que se va a necesitar. Esto ha sido especialmente útil para planificar el tiempo empleado en cada tarea \cite{misc:zenhubestimate}.

También se ha utilizado esta herramienta para extraer los \textit{gráficos burndown} de cada sprint presentes en el anexo.

\subsubsection{GitKraken}

\begin{itemize}
	\tightlist
	\item
	Herramientas consideradas:
	\href{https://www.gitkraken.com/}{GitKraken}, 
	\href{https://desktop.github.com/}{Github Desktop},
	\href{https://www.sourcetreeapp.com/}{Sourcetree}.
	\item
	Herramienta elegida:
	\href{https://www.gitkraken.com/}{GitKraken}.
\end{itemize}

\foothref{GitKraken}{https://www.gitkraken.com/} es un cliente con interfaz gráfica para gestión de proyecto \textit{Git} para Windows, Mac y Linux. Nos permite realizar todas las acciones posibles con git, pero de manera mucho más sencilla y visual.

Permite sincronizarse con Github y otros servicios de alojamiento de repositorios de forma automática, de forma que el contenido del repositorio local y el remoto sea consistente.

\subsubsection{Codacy}

\begin{itemize}
	\tightlist
	\item
	Herramientas consideradas:
	\href{https://www.sonarqube.org/}{SonarQube}, 
	\href{https://codeclimate.com/}{Code Climate},
	\href{https://www.codacy.com/}{Codacy}.
	\item
	Herramienta elegida:
	\href{https://www.codacy.com/}{Codacy}.
\end{itemize}

\foothref{Codacy}{https://www.codacy.com/} es una herramienta de análisis de calidad de código que ayuda a los desarrolladores a realizar código de más rápido y de mayor calidad.

Tiene métricas para medir la complegidad ciclomática, código duplicado, cubrimiento por tests y estadísticas para cada commit o pull request \cite{misc:codacygithub}.

Durante el desarrollo de este proyecto se ha intentado mantener siempre una Certificación A en Codacy con todas las métricas al 100\% para asegurar la calidad del código.

\imagencite{herramientas/codacy}{Certificado de Codacy del proyecto}{misc:codacycertificado}

\subsection {Herramientas de documentación}

\subsubsection{LaTeX}

\begin{itemize}
	\tightlist
	\item
	Herramientas consideradas:
	\href{https://products.office.com/es-es/word}{Microsoft Word}, 
	\href{https://www.latex-project.org/}{LaTeX},
	\href{https://es.libreoffice.org/}{LibreOffice},
	\href{https://www.openoffice.org/es/}{OpenOffice}.
	\item
	Herramienta elegida:
	\foothref{LaTeX}{https://www.latex-project.org/}.
\end{itemize}

Esta memoria está escrita en \LaTeX. Se trata de un sistema de preparación de documentos para la creación de documentos con una alta calidad tipográfica. Se usa frecuentemente en la escritura de libros y artículos científicos, entre otra cosas por la facilidad de incluir expresiones matemáticas \cite{wiki:latex} \cite{wiki:latexwikibooks}.

Para utilizar esta herramienta se ha utilizado \foothref{TeXstudio}{https://www.texstudio.org/}. Es un editor gratuito y multiplataforma para editar documentos escritos en Latex. Incluye iconos para localizar los comandos más comunes, corrector ortográfico, visor de PDF y descarga los paquetes de forma automática.

Este documento se ha escrito utilizando la siguiente plantilla de Latex~\cite{misc:plantillalatex}.

\subsubsection{StarUML}

\begin{itemize}
	\tightlist
	\item
	Herramientas consideradas:
	\href{http://staruml.io/}{StarUML}, 
	\href{http://argouml.tigris.org/}{ArgoUML},
	\href{http://astah.net/}{Astah}.
	\item
	Herramienta elegida:
	\href{http://staruml.io/}{StarUML}.
\end{itemize}

\foothref{StarUML}{http://staruml.io/} es una herramienta para modelar diagramas UML que nos permite crear diagramas de clases, objetos, casos de uso, componente, estructurales, de comunicación, de estado y de actividad y entidad-relación.

Funciona de manera gráfica, arrastrando objetos y conectándolos entre ellos permitiendo añadir atributos tanto a los objetos como a las relaciones.
