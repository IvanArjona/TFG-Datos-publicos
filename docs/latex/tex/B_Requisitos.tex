\apendice{Especificación de Requisitos}

\section{Introducción}

En este apéndice se describirán los objetivos de la aplicación y se detallarán tanto los requisitos funcionales como los no funcionales.

\section{Objetivos generales}

\begin{itemize}
	\item Integrar varias fuentes de datos públicas en una única base de datos. Estos datos son datos de carácter sociológicos, económicos y demográfico a nivel municipal en España.
	\item Permitir añadir nuevos conjunto de datos de forma sencilla.
	\item Desarrollar un algoritmo para la carga de varias fuentes de datos en una base de datos de manera automatizada.
	\item Desarrollo de una aplicación web que permita la consulta de los datos de manera sencilla y visual.
	\item Facilitar la interpretación de los datos utilizando un mapas coropléticos interactivo.
	\item Desplegar la aplicación web en un servidor de forma que sea fácil de actualizar cada vez que se realice un cambio. Además de funcionar en un entorno local.
\end{itemize}

\section{Catalogo de requisitos}

\subsection{Requisitos funcionales}

\begin{itemize}
	\item \textbf{RF-1 Cargar datos}: El administrador de datos debe poder cargar y actualizar los datos desde sus respectivas fuentes de forma automatizada.
	\item \textbf{RF-2 Consulta}: Los usuarios deben poder consultar información de las fuentes de datos.
	\begin{itemize}
		\item \textbf{RF-2.1}: Podrán realizarse varias subconsultas al mismo tiempo.
		\item \textbf{RF-2.2}: Se podrá seleccionar las columnas resultantes a mostrar.
		\item \textbf{RF-2.3}: Se podrá filtrar según los campos de una de las fuentes de datos.
	\end{itemize}
	\item \textbf{RF-3 Visualizar datos}: Los usuarios deben poder visualizar los datos de una consulta en forma de mapa coroplético.
	\begin{itemize}
		\item \textbf{RF-3.1}: Los datos podrán elegirse de cualquier columna numérica.
		\item \textbf{RF-3.2}: Se podrán mostrar los datos en un mapa a nivel municipal o provincial.
		\item \textbf{RF-3.3}: Los datos de una columna se agregarán utilizando varios métodos (media, suma y cuenta).
	\end{itemize}
	\item \textbf{RF-4 Exportar}: Los usuarios deben poder exportar datos en varios formatos.
	\item \textbf{RF-5 Juntar csv}: Los usuarios deben poder juntar varios archivos csv.
		\begin{itemize}
		\item \textbf{RF-5.1}: Juntar los archivos mediante una columna común.
		\item \textbf{RF-5.2}: Utilizando varios tipos de join (\textit{inner}, \textit{outer}, \textit{left}, \textit{right}).
		\item \textbf{RF-5.3}: Seleccionar la ruta donde se exporta el resultado.
	\end{itemize}
\end{itemize}

\subsection{Requisitos no funcionales}

\begin{itemize}
	\item \textbf{RNF-1 Usabilidad}: La aplicación debe ser fácil de usar e intuitiva para el usuario.
	\item \textbf{RNF-2 Rendimiento}: La aplicación debe cargar en un tiempo aceptable.
	\item \textbf{RNF-3 Mantenimibilidad}: La aplicación debe permitir añadir características de forma sencilla.
	\item \textbf{RNF-4 Compatibilidad}: La aplicación debe funciona correctamente en los navegadores modernos más utilizados (Edge, Chrome, Firefox, Opera y Safari).
	\item \textbf{RNF-5 Responsividad}: La aplicación debe funcionar en pantallas de cualquier tamaño y adaptar su interfaz a cada pantalla.
	\item \textbf{RNF-6 Escalabilidad}: La aplicación debe poder aumentar su rendimiento al aumentar recursos hardware.
	\item \textbf{RNF-7 Facilidad de despliegue}: La aplicación debe poder desplegarse en un servidor de forma sencilla.
	\item \textbf{RNF-8 Software libre}: Utilizar software libre siempre que sea posible.
\end{itemize}

\section{Especificación de requisitos}

En esta sección se desarrollan los casos de uso relacionados con los requisitos funcionales del apartado anterior, se enumeran los actores que interactúan con la aplicación y se incluye el diagrama de casos de uso.

\subsection{Descripción de casos de uso}

A continuación se desarrolla la tabla de cada uno de los casos de uso.
\newpage

% Caso de uso 1

\casoDeUso{1}{Carga de datos}
{RF-1}
{Carga el contenido de las fuentes de datos a la base de datos de la aplicación.}
{Se ha iniciado la base de datos}
{
	\item El administrador de datos ejecuta el script para cargar las fuentes de datos.
}
{La información de las fuentes de datos se cargan en la base de datos.}
{
	\item Alguna fuente no está disponible.
	\item La estructura de los datos de alguna fuente ha cambiado.
}
{Alta}

% Caso de uso 2

\casoDeUso{2}{Consulta de datos}
{RF-2, RF-2.1, RF-2.2, RF-2.3}
{Permite al usuario consultar una o varias fuentes de datos.}
{Los datos están cargados en la base de datos.}
{
	\item El usuario entra en la página de consulta.
	\item Selecciona la fuente de datos.
	\item Selecciona un filtro (columna, comparador y valor).
	\item Si quiere añadir más fuentes pulsa "+" y repite desde el paso 2.
	\item Selecciona el tipo de \textit{join} (\textit{inner}, \textit{outer}, \textit{left}, \textit{right})
	\item Escoge las columnas a mostrar y el número de filas.
	\item Pulsa el botón consultar.
}
{Se muestra la consulta realizada por el usuario.}
{
	\item No hay ningún dato que concuerde con el filtro.
	\item El número de filas totales es menor que el número a mostrar (avisa al usuario).
}
{Alta}

% Caso de uso 3

\casoDeUso{3}{Visualización de datos}
{RF-3, RF-3.1, RF-3.2, RF-3.3}
{Permite al usuario visualizar un conjunto de datos en el mapa.}
{Se ha realizado una consulta (CU-2 \ref{CU:2}).}
{
	\item El usuario realiza una consulta.
	\item Selecciona el método de agregación (mean, sum, count).
	\item Selecciona el nivel al que mostrar el mapa (Municipios o provincias).
	\item Escoge la columna de datos a representar.
}
{Se muestra un mapa coroplético con los datos de la columna seleccionada.}
{
	\item Error al cargar las divisiones geográficas.
	\item No hay datos suficientes.
}
{Media}

% Caso de uso 4

\casoDeUso{4}{Exportar consulta}
{RF-4}
{Exporta el resultado de una consulta en varios formatos.}
{Se ha realizado una consulta (CU-2 \ref{CU:2}).}
{
	\item El usuario realiza una consulta.
	\item Pulsa el botón correspondiente al formato en el que quiere descargar el resultado de la consulta.
}
{Descarga un archivo con los datos de la consulta en el formato seleccionado.}
{
	\item No hay datos suficientes.
}
{Media}

% Caso de uso 5

\casoDeUso{5}{Juntar ficheros csv}
{RF-5, RF-5.1, RF-5.2, RF-5.3}
{Junta dos o más ficheros csv en uno solo.}
{Se tienen al menos dos ficheros csv con al menos una columna común.}
{
	\item El usuario abre la aplicación para juntar csv.
	\item Sube al menos dos ficheros csv.
	\item Selecciona la columna por la que juntarlos.
	\item Selecciona el tipo de \textit{join} (\textit{inner}, \textit{outer}, \textit{left}, \textit{right})
	\item Pulsa el botón `Join'.
	\item Seleccionar la ruta donde guardar el archivo resultante.
}
{Descarga un único fichero csv con la combinación de los anteriores en la ruta seleccionada.}
{
	\item Hay menos de dos ficheros.
	\item No tienen columnas en común.
	\item No se puede acceder a alguno de los ficheros.
	\item Alguno de los archivos no está en formato csv.
	\item No se puede escribir en la ruta seleccionada.
}
{Media}

\subsection{Diagrama de casos de uso}

\imagen{diagramas/casos-de-uso}{Diagrama de casos de uso}
